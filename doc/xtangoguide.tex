\documentstyle[]{article}

\title{The XTANGO Environment and Differences from TANGO}
\author{Doug Hayes}
\date{3 November 1990}

\textheight 9in
\textwidth 6.5in
\topmargin -0.5in
\oddsidemargin 0.0in
\evensidemargin 0.0in

\begin{document}

\maketitle

\section{Running an Existing Animation}

To run an XTANGO animation, just type the animation's name at the shell
prompt.  A window frame will appear and, after positioning the window frame,
a blank animation window with various pushbuttons will appear---most
notably a pushbutton in the lower left-hand corner named {\bf run animation}.
Select this pushbutton to display the animation.

At any time before starting the animation with the {\bf run animation}
pushbutton or while the animation is running, the other pushbuttons may be
selected.  These pushbuttons affect what parts of an animation is
displayed (the arrows and zoom pushbuttons labeled {\bf in} and {\bf out}
along the left-hand edge) or the way XTANGO behaves (the pushbuttons
along the bottom edge and the scrollbar along the right-hand edge).

An animation may be aborted at any time by selecting the {\bf quit}
pushbutton in the lower right-hand corner of the animation window.

\subsection{Panning and Zooming}

The pushbuttons along the left-hand edge of the animation screen affect
what portions of an animation are displayed while an animation is running.
To pan the animation window in a specific direction, select the
appropriate arrow pushbutton.  To zoom in on a section of the animation
select the {\bf in} pushbutton; select the {\bf out} pushbutton to zoom back
out.

\subsection{Pausing a Running Animation}

After initially selecting the {\bf run animation} pushbutton to start the
animation running, the {\bf run animation} pushbutton changes its name to
{\bf pause}.  An animation may be paused at any time by selecting the
{\bf pause} pushbutton which then pauses the animation and changes its name
to {\bf unpause.} To continue the animation, select the {\bf unpause}
pushbutton which then changes its name back to {\bf pause}.

\subsection{Changing the Animation Refresh Mode}

XTANGO currently supports three different times to redraw all images
displayed in an animation: between each frame, between each scene, and
none.  The default refresh mode is between each {\em frame}.

There is a direct trade-off between the frequency of redrawing and the
speed of the animation.  Refreshing {\bf by frame} produces a smooth
animation but runs the slowest.  On the other hand, {\bf no refresh}
produces a quick animation possibly with many holes in the images
(created when one image passes over another).  Try the {\bf no refresh}
mode when initially building an animation.  Then switch to {\bf by scene}
to iron out the minor bugs.  Finally use the {\bf by frame} mode when an
animation is complete.

To change the refresh mode, select the {\bf mode} pushbutton either
before running an animation or during the animation.  A menu will
appear containing the three choices and the current choice will be
indicated with a checkmark.  After selecting the appropriate choice,
the new refresh mode will take effect immediately.

Select the {\bf refresh} pushbutton located next to the {\bf quit}
pushbutton to force a refresh at any time.

\subsection{Slowing Down a Running Animation}

Use the vertical scrollbar located along the right-hand edge of the
animation window to control the amount of time between animation frames.
The animation runs fastest when the scrollbar's ``thumb'' (the black,
movable rectangle) is positioned at the top of the scrollbar; the
animation becomes progressively slower as the thumb is moved towards
the bottom of the scrollbar.

To change the position of the thumb, either select the thumb and drag
it to a new location, select a location within the scrollbar (and the
thumb will jump there), or select one of the arrows located at the ends
of the scrollbar (and the thumb will step in the desired direction).

\section{Using {\tt TangoAlgoOp}}

XTANGO differs from the original TANGO in only a few areas.  Most
noticeably the algorithm to animate, the animation scenes, and the XTANGO
engine are bound together at link time to create a stand-alone binary.
Contrast this with the way the original TANGO is set up.  There, the algorithm
to animate runs as a separate process which sends messages to another process,
the TANGO engine.  The TANGO engine then dynamically loads animation
scenes only as needed.

In an attempt to keep the two environments as similar as possible we created
a new mechanism, {\tt TANGOalgoOp}, to simulate the message-passing mechanism
in TANGO, {\tt MSGsenda}.  Whenever {\tt MSGsenda} would be used for TANGO,
{\tt TANGOalgoOp} should be used for XTANGO.

The purpose of {\tt TANGOalgoOp} is to alias animation scene functions with a
string and to be able to call a list of scene functions.  To do this
{\tt TANGOalgOp} requires a little help to map between strings and function
names by using the {\tt NAME\_FUNCT} structure.

The {\tt NAME\_FUNCT} structure contains a list of string name aliases and
a list of animation scene functions to call whenever the alias string is
used.  This structure {\em must appear as the first argument}\/ to
{\tt TANGOalgoOp} unlike TANGO's {\tt MSGsenda}.  The easiest way to set up a
{\tt NAME\_FUNCT} structure is to initialize it staticly at the top of the
program containing the algorithm to animate.  For example, this
{\tt NAME\_FUNCT} was taken from ``bpack.c:''
\begin{tt}
\begin{verbatim}

   void AnimInit(), AnimNewWeight(), AnimMoveTo(), AnimInPlace();

   static NAME_FUNCT fcn[] = { {"Init",       1,  {VOID, (FPTR)AnimInit}},
                               {"NewWeight",  1,  {VOID, (FPTR)AnimNewWeight}},
                               {"Failure",    1,  {VOID, (FPTR)AnimMoveTo}},
                               {"Success",    2, {{VOID, (FPTR)AnimMoveTo},
                                                  {VOID, (FPTR)AnimInPlace}}},
                               {NULL,         0,   NULL,       NULL} };

\end{verbatim}
\end{tt}
Notice the list of scene functions to call when ``Success'' is used in
{\tt TANGOalgoOp}.  Also notice the {\tt NULL}s used to terminate the
structure.

Here is an example of using {\tt TANGOalgoOp}, also taken from ``bpack.c:''
\begin{tt}
\begin{verbatim}

   TANGOalgoOp(fcn, "Init", bin, numBins);

\end{verbatim}
\end{tt}

There are two special string aliases used by XTANGO which are not declared in
the {\tt NAME\_FUNCT} structure explicitly: ``BEGIN'' and ``END.''  A call to
``BEGIN'' must be included {\em before} calling other string aliases because
``BEGIN'' initializes XTANGO.  Place the call to ``END'' as the last function
in your algorithm animation routine (usually {\tt main()}).  The ``END'' routine
makes sure all animation scenes have been shown and waits until the {\bf quit}
pushbutton is selected to end the animation.

The best way to understand how this mechanism works is to look at a few
examples.  Browse through the {\bf anims} subdirectory to get a better feel
for {\tt TANGOalgoOp} and its usage.

\end{document}
