\documentstyle[]{article}

\textheight 9in
\textwidth 6.5in
\topmargin -0.25in
\oddsidemargin 0.0in
\evensidemargin 0.0in

\begin{document}

\markright{XTANGO doc}
\pagestyle{myheadings}
\thispagestyle{empty}

\begin{center}
   \Large\bf  XTANGO Algorithm Animation Designer's Package\\[.2in]
\large  John T. Stasko                 \\
        Doug Hayes               \\[.2in]
        College of Computing \\
        Georgia Institute of Technology\\
        Atlanta, GA  30332-0280     \\[.2in]
        stasko@cc.gatech.edu        \\[.1in]
        October 21, 1992
\end{center}
This package is an implementation of the {\em path-transition}
algorithm animation framework developed by John Stasko.  (XTANGO =
X-window Transition-based ANimation GeneratiOn.)  It is designed to
allow users to easily animate an algorithm or program without having
to write any low-level graphics code.  While the framework contains
extensible models for images and transitions, in this package we have
implemented ten types of images (lines, rectangles, circles, ellipses,
polylines, polygons, splines, bitmaps, text, and composites) and ten
types of transitions (fill, visibility, color, move, resize, zoom, raise,
grab, lower, delete, refresh and delay). These primitives give us a
compact working set for easy learning, yet they still provide
sufficient depth to generate moderately complex animation sequences.

\section{Data objects}
The data types provided by this package are
\begin{itemize}
\item TANGO\_LOC\\
This is the handle that identifies a particular position in the viewing
window.  Locations are defined by real-valued ({\em x}, {\em y}) coordinate
pairs.
\item TANGO\_IMAGE\\
This is the handle that identifies a particular graphical image on the
screen.  In this implementation an image can be a line, rectangle,
circle, ellipse, polyline, polygon, spline, bitmap, text, or a
composite of the previous types.
\item TANGO\_PATH\\
This is the handle that identifies an animation path along which an image
is modified.  A path can be thought of as a list of relative
offsets, each relative to the previous position.
\item TANGO\_TRANS\\
This is the handle that identifies an actual transition action.  A transition
action is comprised of 1)the actual type of the transition such as fill
style change, visibility change, color change, motion, resizing, and so forth,
or some combination thereof, 2)the image to which it applies, and 3)a path to
``modify'' along.
\end{itemize}

\section{Entry Points}
\subsection{Algorithm Operations}
TANGO contains one call that passes information from a program to its
accompanying animation.  This call should be placed at key points
within the program being animated.

\vspace{1em}
\begin{verbatim}
char *
TANGOalgoOp (map,name,p1,p2,p3..)
   NAME_FUNCT map[];
   char *name;
   params p1,p2,p3...;
\end{verbatim}
This routine sends an algorithm operation named {\em name} to the
animation controller.  The structures within {\em map} define what functions
should be called for the given operation name.  The trailing
parameters are passed to each target function in the order they are
provided.  

{\em map} consists of an arbitrary number of fields.  Each filed has 3
components.  The first is an algorithm operation string name.  This
defines which operation is being referenced.  The second component is
the number of trailing functions to be called for this operation.  The
final component can be made up of subcomponents.  Each has 2 parts.
The first is the return value of the function (should be VOID or
CHAR\_PTR).  The second is a function pointer to an actual animation
scene to be called.  The number of subcomponents in this component
should match the second component of the entire field.  The definition
for a {\tt NAME\_FUNCT} is provided in xtango.h.

To see how to set up a {\em map} structure, examine some of the
existing animations in the xtango/anims directory.  Note, the call
TANGOalgoOp(map, "BEGIN") should always be the first statement of your
driver program and the call TANGOalgoOp(map, "END") should always be
the last. 

\subsection{Initialization/Termination}
The animation package creates its own graphical display environment on top 
of the X11 Window system.

Animation window graphics are laid out in an infinite coordinate system with
positive {\em x} values moving right and positive {\em y} values moving down.
To start, the upper left corner is (0.0,0.0) and the lower right corner is
(1.0,1.0).  Locations outside the predefined range can be accessed; images
displayed there simply will not be seen until those coordinates are scrolled
into the viewing area.

\vspace{1em}
\begin{verbatim}
void
TANGOinit ( )
\end{verbatim}
This routine is used to initialize the TANGO package.  It will be called
automatically by TANGOalgoOp(mappingstruct, "BEGIN").

\vspace{1em}
\begin{verbatim}
void
TANGOend ( )
\end{verbatim}
This routine is used to terminate the TANGO package.  It will be called
automatically by TANGOalgoOp(mappingstruct, "END").

\subsection{Locations}
The ability to save various logical coordinate points in the viewing window
is a powerful tool.  Because locations are identified within a real-valued
coordinate system, the same animation can be performed in windows of varying
physical size without changing any of the code controlling the animation.

\vspace{1em}
\begin{verbatim}
TANGO_LOC
TANGOloc_create (x,y) 
   double x,y;
\end{verbatim}
This routine creates a logical location with the given {\em x} and
{\em y}
coordinates.

\vspace{1em}
\begin{verbatim}
double
TANGOloc_X (loc) 
   TANGO_LOC loc;
\end{verbatim}
This routine returns the {\em x} coordinate of the given TANGO\_LOC.

\vspace{1em}
\begin{verbatim}
double
TANGOloc_Y (loc) 
   TANGO_LOC loc;
\end{verbatim}
This routine returns the {\em y} coordinate of the given TANGO\_LOC.

\vspace{1em}
\begin{verbatim}
void
TANGOloc_inquire (loc,x,y) 
   TANGO_LOC loc;
   double *x,*y;
\end{verbatim}
This routine returns, through its parameters, the {\em x} and {\em y}
values represented by the given TANGO\_LOC.

\vspace{1em}
\begin{verbatim}
TANGO_LOC
TANGOloc_modify (loc,x,y) 
   TANGO_LOC loc;
   double x,y;
\end{verbatim}
This routine creates a new TANGO\_LOC that has coordinates equal to the given
location plus the parameter values provided as arguments. (The values are
simply added together).

\vspace{1em}
\begin{verbatim}
int
TANGOloc_equal (loc1,loc2) 
   TANGO_LOC loc1,loc2;
\end{verbatim}
This routine returns 1 if the two locations have the same {\em x}
and {\em y} coordinates (within a very small error factor) and 0 otherwise.

\subsection{Images}
Nine simple graphical objects (lines, rectangles, circles, ellipses,
polygons, polylines, splines, bitmaps,
and text) are the basis of the animation package.  Rectangles, circles,
ellipses, and polygons can have varying fill styles.  Lines can have different
thicknesses and styles.  Currently, this base set is limited to provide a
simpler basis to design animations.

\vspace{1em}
\begin{verbatim}
TANGO_IMAGE
TANGOimage_create (type,p1,p2,...,pn) 
   TANGO_IMAGE_TYPE type;
   param p1,p2,...,pn;
\end{verbatim}
This routine allocates a graphical image and returns a handle to it.
{\bf Important:} The image will not become visible until the first frame of a
subsequent transition occurs, however.  The image {\em type} can be
one of the following types: TANGO\_IMAGE\_TYPE\_LINE,
TANGO\_IMAGE\_TYPE\_RECTANGLE,\\
 TANGO\_IMAGE\_TYPE\_CIRCLE,
TANGO\_IMAGE\_TYPE\_ELLIPSE, TANGO\_IMAGE\_TYPE\_POLYLINE,
TANGO\_IMAGE\_TYPE\_POLYGON, TANGO\_IMAGE\_TYPE\_SPLINE,
TANGO\_IMAGE\_TYPE\_TEXT,\\
TANGO\_IMAGE\_TYPE\_BITMAP, or TANGO\_IMAGE\_TYPE\_COMPOSITE.  Depending
upon the image's type, the subsequent parameters must follow the
appropriate protocol (TC stands for TANGO\_COLOR below):

LINE - (lx,ly,vis,color,sx,sy,wid,sty,arrow) double
lx,ly,sx,sy,wid,sty; TC color; int vis,arrow;\\ 
The {\em lx} and {\em ly}
parameters correspond to locations, and the {\em sx} and {\em sy}
parameters correspond to sizes.  Line sizes can be positive or
negative.  {\em wid} defines the width of the line; it can range from
0.0 to 1.0 (corresponding roughly to percentages).  0.0 corresponds to
a thin line, and 1.0 corresponds to the thickest line.  Currently, we
have implemented three thicknesses that are achieved by values in the
range 0.0 to 0.333, 0.333 to 0.667, and 0.667 to 1.0, respectively.
{\em sty} defines the line's style.  It ranges from 0.0 to 1.0 also.
We have implemented three styles: 0.0 to 0.333 defines a dotted line,
0.333 to 0.667 defines a dashed line, and 0.667 to 1.0 defines a solid
line.  {\em color} gives the color of the image.  This can be one of
TANGO\_COLOR\_WHITE, TANGO\_COLOR\_BLACK, TANGO\_COLOR\_RED,
TANGO\_COLOR\_ORANGE,\\
 TANGO\_COLOR\_YELLOW, TANGO\_COLOR\_GREEN,
TANGO\_COLOR\_BLUE, or\\
 TANGO\_COLOR\_MAROON, or it can be a
TANGO\_COLOR created via the call TANGOload\_color. If the {\em vis}
parameter is 1, the image is initially visible.  If {\em vis} is 0,
the image is initially invisible.  {\em arrow} designates an arrow
style: 0 means no arrows, 1 means a forward arrow, 2 means a backward
arrow, and 3 means a bi-directional arrow.

RECTANGLE - (lx,ly,vis,color,sx,sy,fill) double lx,ly,sx,sy,fill; int
vis; TC color;\\
Rectangles are located by their upper left corner, with only positive sizes
ranging down and to the right allowed.  The {\em fill} parameter is a value
between 0.0 and 1.0.  0.0 corresponds to an unfilled outline and 1.0
corresponds  to 100 per cent fill in the given color.  (Forty graduated fill
styles are actually implemented.)

CIRCLE - (lx,ly,vis,color,radius,fill) double lx,ly,radius,fill; int
vis; TC color;\\
For circles, the location pair denotes the center of the circle, with the
{\em radius} parameter defining the radius of the circle.

ELLIPSE - (lx,ly,vis,color,rx,ry,fill) double lx,ly,rx,ry,fill; int
vis; TC color;\\
For ellipses, the location pair identifies the ellipse's center.  The
{\em rx}
and {\em ry} values denote the ellispe's radii in {\em x} and {\em y}
repspectively.

POLYLINE - (lx,ly,vis,clr,vert,vx,vy,wid,sty,arw; double
lx,ly,vx[],vy[],wid,sty; int vis,vert,arw; TC clr;\\
The location pair identifies the beginning vertex of the polyline.  The value
{\em vert} identifies the number of vertices on the polyline.
There can be a maximum of 8.  The arrays {\em vx} and {\em vy} define the
{\em x} and {\em y} {\bf relative} offsets of the other polyline vertices from
the location vertex given earlier, not from each other.  The vertex identified
by the {\em lx, ly} pair should not be included in these arrays, i.e., the
arrays can be at most 7 elements.  {\em wid} and {\em sty} define the width
and style of the polyline segments just as in the line image type.
{\em arrow} works just as in a line. In a polyline, the arrow's forward
direction is determined by direction of the last polyline edge that is
non-negative (vice-versa for backward).

POLYGON - (lx,ly,vis,color,sides,vx,vy,fill; double lx,ly,vx[],vy[],fill; int
vis,sides; TC color;\\
The location pair identifies a vertex of the polygon.  The value {\em sides}
identifies the number of sides (or equivalently vertices) on the polygon.
There can be a maximum of 8.  The arrays {\em vx} and {\em vy} define the
{\em x} and {\em y} {\bf relative} offsets of the other polygon vertices from
the location vertex given earlier.  The vertex identified by the {\em lx, ly}
pair should not be included in these arrays, i.e., the arrays can be at most 7
elements.  Polygons can be concave or convex, but they should not be
self-intersecting.  {\em fill} gives the fill value of the polygon.

SPLINE - (lx,ly,vis,color,vert,vx,vy,wid,sty; double lx,ly,vx[],vy[],wid,sty;
int vis,vert; TC color\\
The location pair identifies the beginning control point of the spline.  The
value {\em vert} identifies the number of points on the spline.
There can be a maximum of 8.  The arrays {\em vx} and {\em vy} define the
{\em x} and {\em y} {\bf relative} offsets of the other spline points from
the location point given earlier.  The point identified by the {\em lx, ly}
pair should not be included in these arrays, i.e., the arrays can be at most 7
elements.  {\em wid} and {\em sty} define the width and style of the spline
just as in the line image type.

BITMAP - (lx,ly,vis,bmaps,num,wid,hei) double lx,ly; int vis,nu,wid,hei; int *bmaps;\\
A bitmap image is really a series of 2-d bitmaps.  The location pair
defines the top-left position of the image.  {\em bmaps} is a 3-d integer
array in which the first dimension is the number of bitmaps, second is
the width of each bitmap, and third is the height of each bitmap.
Each position in the array specifies a pixel color.  The integers
placed there correspond to the integer values for TANGO\_COLORs.  For
example, the bitmap definition below shows a yellow line swing down
across a black background.  A bitmap begins showing the first 2-d
bitmap array defined for it.  (The SHUFFLE transition is useful to
then cycle through the others.)

\begin{verbatim}

static int bmaps[3][4][4]= {
         {{1,1,1,1},
          {7,7,7,7},
          {7,7,7,7},
          {7,7,7,7}},

         {{7,7,7,1},
          {7,7,1,7},
          {7,1,7,7},
          {1,7,7,7}},

         {{7,7,7,1},
          {7,7,7,1},
          {7,7,7,1},
          {7,7,7,1}}};

\end{verbatim}

TEXT - (lx,ly,vis,color,fname,str,ctr) double lx,ly; TC color; int vis,ctr;
char *str, *fname;\\
The location pair identifies the lower left corner of where the text will be
placed if the {\em ctr} parameter is 0, or the center location of where
the text will be placed if the parameter is 1.  The {\em fname} parameter
tells what font should be used (if NULL, the default font is used), and the
{\em str} parameter gives the text to be displayed.

COMPOSITE - (lx,ly,vis,im) double lx,ly; int vis; TANGO\_IMAGE\_COMPONENT im[];\\
This option allows a composite image to be loaded.  A composite image
is one that is made up of primary subimages (all above) and that has
its  own location and visibility.  The {\em im} argument should be an
array of TANGO\_IMAGE\_COMPONENT objects.  Each of these has the form

\begin{verbatim}

        typedef struct _IMAGE_COMP {
              TANGO_IMAGE_TYPE type;
              char *args;
        } TANGO_IMAGE_COMPONENT;

\end{verbatim}

{\em type} is the type of the subimage.  The {\em args} parameter is a
string of values encoded as follows (note, no commas):

For line: "lx ly color sx sy wid sty arrow"

For rectangle: "lx ly color sx sy fill"

For circle: "lx ly color rad fill"

For ellipse: "lx lx color radx rady fill"

For polyline: "lx ly color wid sty arrow vertices vx1 vy1 vx2 vy2 ..."

For polygon: "lx ly color fill sides vx1 vy1 vx2 vy2 ..."

For spline: "lx ly color wid sty vertices vx1 vy1 vx2 vy2 ..."

For text: "lx ly color fontname str centered"

Each line defines a subimage in the composite image.  The lx and ly values
are {\em relative} locations with respect to the location given in the
TANGOimage\_create call.  The color should be a color written out.  The other
parameters are of the same type and are used just like the above parameters
for the basic image types.  Note that bitmap subimages are not allowed
in composites.  The following example will define a door.

\begin{verbatim}

static TANGO_IMAGE_COMPONENT door[5] = {
   { TANGO_IMAGE_TYPE_RECTANGLE, "0.0 0.0 TANGO_COLOR_BLACK 0.1 0.2 0.0"},
   { TANGO_IMAGE_TYPE_RECTANGLE, "0.01 0.01 TANGO_COLOR_BLACK 0.08 0.08 0.0"},
   { TANGO_IMAGE_TYPE_RECTANGLE, "0.01 0.11 TANGO_COLOR_BLACK 0.08 0.08 0.0"},
   { TANGO_IMAGE_TYPE_CIRCLE, "0.09 0.1 TANGO_COLOR_BLACK 0.005 0.0"},
   { TANGO_IMAGE_TYPE_COMPOSITE, NULL}
};

\end{verbatim}

Note that the array definition should always end with a COMPOSITE type that
tells TANGO that this is the end of the definition.

Composite images are useful when a certain group of images (such as that of a
stick person) is to be created and used multiple times in an animation.  When
an inquiry is made about the location of a composite image, the current
location of the entire composite image is returned.  Currently, transitions
applied to a composite image will affect all the subimages in the same
way. Ideally, this shouldn't happen, i.e., when we resize a person, we want
the relative positions of the components to remain constant.

\vspace{1em}
\begin{verbatim}
TANGO_IMAGE
TANGOimage_copy (image) 
   TANGO_IMAGE image;
\end{verbatim}
This routine returns a new TANGO\_IMAGE which is an exact copy of the
one given as a parameter (at that instant in time).  Just like
TANGOimage\_create, the new image will not appear until a transition is
performed. 

\vspace{1em}
\begin{verbatim}
TANGO_LOC
TANGOimage_loc (image,part) 
   TANGO_IMAGE image;
   TANGO_PART_TYPE part;
\end{verbatim}
This routine returns an TANGO\_LOC that corresponds to the location of
the given part of the given {\em image. } Valid TANGO\_PART\_TYPES
include TANGO\_PART\_TYPE\_C (center), and the compass directions
TANGO\_PART\_TYPE\_NW, TANGO\_PART\_TYPE\_N, TANGO\_PART\_TYPE\_NE,\\
TANGO\_PART\_TYPE\_E, TANGO\_PART\_TYPE\_SE, TANGO\_PART\_TYPE\_S,\\
TANGO\_PART\_TYPE\_SW, and TANGO\_PART\_TYPE\_W.  For rectangles and
circles, actual locations on the image boundary (other than the center) are
returned.  For lines, ellipses, polylines, polygons, splines, and
text, a bounding box location is returned.  For composites, a bounding
box of all the subimages is returned.

\subsection{Paths}
A path is the graphical basis upon which image modification is based.  It can
be thought of as a list of relative ({\em x} and {\em y}) offsets from the
previous location.  Under the move transition, the control points or offsets
in the path define where the image should next be located.  Think of a control
point in a path like a frame in a motion picture.  In subsequent frames,
images have changed location by some small increment.  If we iterate the
frames in this series, the image appears to move along this path.  The
{\em length} of a path is the number of offsets it contains.

Paths are not only used for movement, however.  Every transition utilizes a
path argument to define its action.

\vspace{1em}
\begin{verbatim}
TANGO_PATH
TANGOpath_create (n,xoffsets,yoffsets) 
   int n;
   double xoffsets[],yoffsets[];
\end{verbatim}
This routine creates a path that has length {\em n,} designated by the given
sequence of {\em x} and {\em y} offset pairs.

\vspace{1em}
\begin{verbatim}
TANGO_PATH
TANGOpath_load (file) 
   char *file;
\end{verbatim}
This routine loads a path from a file in which it has been stored (via
TANGOpath\_store).  Consequently, this routine provides the capability to
reuse paths from session to session.  If the path was unable to be loaded,
the routine returns NULL.

\vspace{1em}
\begin{verbatim}
TANGO_PATH
TANGOpath_store (path,file) 
   TANGO_PATH path;
   char *file;
\end{verbatim}
This routine stores a path into a file for later use.  The path may have been
created in any valid way.  If the path is stored successfully, the routine
returns the same TANGO\_PATH handle.  If the store fails, the routine
returns NULL.

\vspace{1em}
\begin{verbatim}
int
TANGOpath_length (path) 
   TANGO_PATH path;
\end{verbatim}
This routine returns the numbers of offsets in the given path.

\vspace{1em}
\begin{verbatim}
double
TANGOpath_dx (path) 
   TANGO_PATH path;

double
TANGOpath_dy (path)
   TANGO_PATH path;
\end{verbatim}
These routines return the {\em x} or {\em y} distance, respectively, that the
given path traverses from start to finish.

\vspace{1em}
\begin{verbatim}
TANGO_PATH
TANGOpath_type (type) 
   TANGO_PATH_TYPE type;
\end{verbatim}
This routine creates a path which is one of three basic path types.  They
are as follows:

\vspace{1em}

\begin{tabular}{|l|l|}\hline
TANGO\_PATH\_TYPE\_STRAIGHT & move right\\
TANGO\_PATH\_TYPE\_CLOCKWISE & move clockwise right in an upward arc\\
TANGO\_PATH\_TYPE\_COUNTERCLOCKWISE & move counterclockwise left in an
upward arc\\ \hline
\end{tabular}

\vspace{1em}

Each of these paths when created will have 20 relative offset points and
will change in {\em x} by a total value of 0.2 animation coordinate system
units.

The paths created by this routine will probably not be utilized as is,
although there is no reason why they cannot be.  They are provided
primarily as starting paths for use in the subsequent path modification
routines.

\vspace{1em}
\begin{verbatim}
TANGO_PATH
TANGOpath_color (color) 
   TANGO_COLOR color;
\end{verbatim}
This routine returns a one offset path that will change an image to the given
color when the path is used as an argument to an TANGO\_TRANS\_TYPE\_COLOR
transition.  The direction of the offset in the path determines the color to
utilize.  For example, if there is a black rectangle that you wish to be
red, first get a path by calling TANGOpath\_color(TANGO\_COLOR\_RED), then use
the returned path as the third argument to
TANGOtrans\_create(TANGO\_TRANS\_TYPE\_COLOR,image,path). For available colors,
see TANGOimage\_create.  Note that colors created with
TANGOload\_color cannot be used as parameters to this function.

\vspace{1em}
\begin{verbatim}
TANGO_PATH
TANGOpath_add_tail (path,num) 
   TANGO_PATH path;
   int num;

TANGO_PATH
TANGOpath_add_head (path,num)
   TANGO_PATH path;
   int num;
\end{verbatim}
These routines provide ways of elongating a path.  The parameter {\em num}
designates the number of null ($\langle$0.0,0.0$\rangle$) offsets that will be apppended to
the given {\em path} at either the head or tail in order to produce a new path
that is returned.

\vspace{1em}
\begin{verbatim}
TANGO_PATH
TANGOpath_delete_tail (path,num) 
   TANGO_PATH path;
   int num;

TANGO_PATH
TANGOpath_delete_head (path,num) 
   TANGO_PATH path;
   int num;
\end{verbatim}
These routines provide ways of shortening a path.  The parameter {\em num}
designates the number of offsets that will be removed from the head or tail of
the given path in order to produce a new path that is returned.

\vspace{1em}
\begin{verbatim}
TANGO_PATH
TANGOpath_reverse (path) 
   TANGO_PATH path;
\end{verbatim}
This routine returns a path that is the inverse of the given path.
Being an inverse path means that the order of offsets is reversed, and
each offset points in the exact opposite direction.  This routine
provides a way for an image to retrace its tracks from a movement transition.

\vspace{1em}
\begin{verbatim}
TANGO_PATH
TANGOpath_rotate (path,deg) 
   TANGO_PATH path;
   int deg;
\end{verbatim}
This routine returns a path that corresponds to the rotation of the given path
in a counter-clockwise motion.  The number of degrees to rotate is provided by
the integer parameter {\em deg} which should be between 0 and 360.

\vspace{1em}
\begin{verbatim}
TANGO_PATH
TANGOpath_interpolate (path,factor) 
   TANGO_PATH path;
   double factor;
\end{verbatim}
This routine returns a path in which the number of offsets is modified by a
factor given by the {\em factor} parameter.  If a path with length 10 is
interpolated with a factor of 2.0, the returned path will have 20 offsets.  If
the factor is 0.3, the path will have 3 offsets.  Straight linear
interpolation is used.  Consequently, when interpolating a path with choppy,
wavy motion characteristics, individual extrema may be lost with certain
{\em factor} parameters.

\vspace{1em}
\begin{verbatim}
TANGO_PATH
TANGOpath_scale (path,dx,dy) 
   TANGO_PATH path;
   double dx, dy;
\end{verbatim}
This routine returns a path in which each offset is scaled by the given
factors in {\em x} and {\em y}.  That is, each offsets' {\em x}
and {\em y}
values are mulitplied by the {\em dx} and {\em dy} factors, respectively.  So,
for example, to return a path that would be the reflection of a path across an
imaginary vertical line, choose {\em x} = -1.0 and {\em y} = 1.0.

\vspace{1em}
\begin{verbatim}
TANGO_PATH
TANGOpath_extend (path,dx,dy) 
   TANGO_PATH path;
   double dx, dy;
\end{verbatim}
This routine returns a path in which each offset is extended by the given
factors in {\em x} and {\em y}.  That is, each offsets' {\em x}
and {\em y}
values have the {\em dx} and {\em dy} factors added to them, respectively.

\vspace{1em}
\begin{verbatim}
TANGO_PATH
TANGOpath_copy (path) 
   TANGO_PATH path;
\end{verbatim}
This routine returns a new path that has an exact duplicate list of offsets as
the given path.

\vspace{1em}
\begin{verbatim}
TANGO_PATH
TANGOpath_null (num) 
   int num;
\end{verbatim}
This routine provides a way of creating a null path (all $\langle$0.0,0.0$\rangle$ relative
offsets that "go nowhere") that is the given number of offsets long.
This routine is useful for generating paths in which only the number of
offsets is necessary, not their particular values.

\vspace{1em}
\begin{verbatim}
TANGO_PATH
TANGOpath_iterate (path,factor) 
   TANGO_PATH path;
   double factor;
\end{verbatim}
This routine returns a path which is an iteration of a given path.
The {\em factor} parameter provides how many times the given path should be
repeated in the path to be created.

\vspace{1em}
\begin{verbatim}
TANGO_PATH
TANGOpath_concatenate (num,p1,p2,p3,p4,p5,p6,p7,p8,p9,p10) 
   int num;
   TANGO_PATH p1,p2,p3,p4,p5,p6,p7,p8,p9,p10;
\end{verbatim}
This routine creates a path which is the concatenation of some number of
other paths.  The {\em num} parameter provides how many paths are to be
concatenated, and the {\em p}i parameters contain the paths themselves.  The
paths will be concatenated in order from the 1 position on to the num
position. The first offset of a path is relative to the last offset of the
previous path.

\vspace{1em}
\begin{verbatim}
TANGO_PATH
TANGOpath_compose (num,p1,p2,p3,p4,p5,p6,p7,p8,p9,p10) 
   int num;
   TANGO_PATH p1,p2,p3,p4,p5,p6,p7,p8,p9,p10;
\end{verbatim}
This routine returns a path which is the composition of the {\em num} paths
given in the {\em p1} to {\em p}num parameters.  By composition, we mean a
cumulative combination on an offset by offset basis. If a path is thought of
as a vector, the composition of paths produces a new vector that has the same
length as the originals and is like the vector sum of the original
path vectors.  Only paths with an equal number of offsets can be
composed.  Otherwise, the routine returns NULL.

Essentially, path composition takes all the first relative $\langle${\em
x},{\em y}$\rangle$
offsets and adds them together to make a new cumulative first
$\langle${\em x},{\em y}$\rangle$ offset. This is done for each offset in the paths.

\vspace{1em}
\begin{verbatim}
TANGO_PATH
TANGOpath_distance (fromloc,toloc,distance) 
   TANGO_LOC fromloc;
   TANGO_LOC toloc;
   double distance;
\end{verbatim}
This routine returns a path that proceeds in a straight line from the given
{\em fromloc} to the given {\em toloc} with the condition that the path will
have an offset every time the given {\em distance} has been covered.  If the
given distance is larger than the distance between the fromloc and toloc, the
path is only given one offset.  This routine is useful to create paths that
will make images move at the same velocity.

\vspace{1em}
\begin{verbatim}
TANGO_PATH
TANGOpath_example (fromloc,toloc,path) 
   TANGO_LOC fromloc;
   TANGO_LOC toloc;
   TANGO_PATH path;
\end{verbatim}
This routine returns a path which "looks like" the given example path, but
which would move from the logical {\em fromloc} to the logical {\em toloc}.
The created path will have the same number of offsets as the given path. The
general flow of movement in the example path is followed as closely as
possible by maintaining the same ratios between control points in both
paths. Clearly, however, if the two paths move in opposite directions, they
will not look much alike.  Typically, this routine will be used when one wants
an image to end up in some specific position, with the image following some
rough, given trajectory path in order to get there.  If the {\em path}
argument is null, a straight line path of 20 equally spaced offsets is used as
the example path.

\vspace{1em}
\begin{verbatim}
TANGO_PATH
TANGOpath_motion (fromloc,toloc,type) 
   TANGO_LOC fromloc;
   TANGO_LOC toloc;
   TANGO_PATH_TYPE type;
\end{verbatim}
This routine returns a path with movement characteristics of the given
{\em type}, but that also begins at the location with logical {\em x} and
{\em y} coordinates of {\em fromloc} and proceeds to the location with logical
{\em x} and {\em y} coordinates of the {\em toloc}.  Note that the provided
from and to locations need not have absolute screen coordinate meaning.
Because a path is made up of a group of relative offsets, these locations are
provided just as necessary aids in the path creation.  For example, the same
path will be created using from = $\langle$0.2,0.2$\rangle$ and to = $\langle$0.4,0.4$\rangle$ as using from =
$\langle$0.7,0.7$\rangle$ and to = $\langle$0.9,0.9$\rangle$. Most often however, the provided locations will
be specific window coordinates chosen to move a certain image to a specific
location.

The created path will contain 20 offsets and will be modelled to fit the given
path type, which may be one of TANGO\_PATH\_TYPE\_STRAIGHT,
TANGO\_PATH\_TYPE\_CLOCKWISE, or\\
 TANGO\_PATH\_TYPE\_COUNTERCLOCKWISE.  The
straight option creates a path that is a direct line from {\em fromloc} to
{\em toloc} with equally spaced offsets.  The two clock motions will create a
path that curves in the given clock motion direction.

Essentially, this routine is just TANGOpath\_example( fromloc, toloc,
TANGOpath\_type(type) ).

\vspace{1em}
\begin{verbatim}
TANGO_PATH
TANGOpath_smooth (path) 
   TANGO_PATH path;
\end{verbatim}
This routine returns a path which is a ``smoother'' version of the given path.
Essentially, each offsets' value is determined by averaging in the neighboring
offsets' values. Currently, we use 2 times the designated offset plus one
time the previous and subsequent offsets.

\vspace{1em}
\begin{verbatim}
void
TANGOpath_free (num,p1,p2,p3,p4,p5,p6,p7,p8,p9,p10) 
   int num;
   TANGO_PATH p1,p2,p3,p4,p5,p6,p7,p8,p9,p10;
\end{verbatim}
This routine frees the memory space occupied by an TANGO\_PATH.  Any number
from one to ten paths can be passed in, (the {\em num} parameter indicates how
many), and each of the path's memory space will be freed. It is a good idea
to free the space once a path will no longer be used.

\subsection{Transitions}
Transitions are the actual animation actions which affect images on
the screen.  Ten basic types have been implemented in this package:
change in visibility, change in fill style (shade), change in color,
movement, resizing, raising, lowering, grab, delete, refresh, and the
delay action.  These seven basic types can be combined via the complex
transition operators iteration, concatenation, and composition. A
transition is made up of a series or list of frames.  Modifications to
images occur within each frame.

All of the following transition routines which return an TANGO do not
actually perform the transition.  They merely create a logical object which
can later be performed (repeatedly) in order to carry out the desired
action.  Transitions are fully recursive in that compose transitions can be
concatenated, iterate transitions can be composed with ``single'' transitions,
etc.

\vspace{1em}
\begin{verbatim}
TANGO_TRANS
TANGOtrans_create (type,image,path) 
   TANGO_TRANS_TYPE type;
   TANGO_IMAGE image;
   TANGO_PATH path;
\end{verbatim}
This is the routine that creates a transition of one of the basic
types.  The {\em type} parameter should be one of
TANGO\_TRANS\_TYPE\_MOVE, TANGO\_TRANS\_TYPE\_VISIBLE,
TANGO\_TRANS\_TYPE\_FILL, TANGO\_TRANS\_TYPE\_RESIZE,
TANGO\_TRANS\_TYPE\_RESIZE1, TANGO\_TRANS\_TYPE\_RESIZE2,
TANGO\_TRANS\_TYPE\_RESIZE3, TANGO\_TRANS\_TYPE\_RESIZE4,
TANGO\_TRANS\_TYPE\_RESIZE5, TANGO\_TRANS\_TYPE\_RESIZE6,
TANGO\_TRANS\_TYPE\_RESIZE7, TANGO\_TRANS\_TYPE\_GRAB1,
TANGO\_TRANS\_TYPE\_GRAB2, TANGO\_TRANS\_TYPE\_GRAB3,
TANGO\_TRANS\_TYPE\_GRAB4,\\
TANGO\_TRANS\_TYPE\_GRAB5, TANGO\_TRANS\_TYPE\_GRAB6, 
TANGO\_TRANS\_TYPE\_GRAB7,\\
TANGO\_TRANS\_TYPE\_SHUFFLE, TANGO\_TRANS\_TYPE\_RAISE, 
TANGO\_TRANS\_TYPE\_LOWER,\\
TANGO\_TRANS\_TYPE\_DELETE, TANGO\_TRANS\_TYPE\_ZOOM, 
TANGO\_TRANS\_REFRESH, or\\
TANGO\_TRANS\_TYPE\_DELAY.  Individual
transitions  operate as follows:

MOVE - move the given {\em image} along the given {\em path}.  The first
movement of the image corresponds to the first relative offset in the path.
All these relative offsets are with respect to the image's previous position
on the screen.

VISIBLE - switch the visibility of the given {\em image} for each offset in
the given {\em path}.  At each offset in the path, if the image is visible, it
will become invisible, and vice-versa.

COLOR - change the given {\em image} to the color indicated by the
given {\em path}.  See the routine TANGOpath\_color for how to create a
path that will change an image to a certain color.  

FILL - change the "fill" component of the given image.  This works
differently for different types of images.  For rectangles, circles,
ellipses, and polygons, this transition alters the image's fill style
value by the {\em x} value of the offsets in the path.  The {\em x}
value is simply added in.  If the image's fill value goes below 0.0,
it is automatically set back to 0.0.  If it goes above 1.0, it is set
back to 1.0.  Actually, the full range of values between 0.0 and 1.0
is not available.  We currently implement 40 different fills that
range between the two extremes.  For lines, polylines, and splines,
the {\em x} value of each offset is added to the line's {\em width}
parameter value, and the {\em y} value is added to the line's style
parameter value (see TANGOimage\_create for information on width and
styles).  Extreme values are reset to 0.0 and 1.0 as in rectangles and
circles.

RESIZE - resize the given {\em image} along the given {\em path}.
The various types of images each have a ``method'' in which they are resized.
Since lines can have positive or negative sizes, they are resized by altering
the line's size for each offset in the path. Rectangles can have only positive
sizes, so resizing a rectangle corresponds to ``dragging'' the lower right
corner of the rectangle along the given path. If one of the rectangle's
dimensions would become negative, it is set to 0. Circles are resized by
modifying the circle's radius by the amount given in the {\em x} component of
each offset in the path.  On an ellipse, the resize transition adds the
{\em x} value to the ellipse's {\em x} radius and the {\em y} value to the
{\em y} radius.  For polylines, polygons, and splines the transitions
RESIZE1-RESIZE7 modify relative positions of the respective vertex number,
plus all others after it in numerical order, by the relative offsets of the
path.  These transitions are useful, for example, with a forward arrow
polyline that has many of its edges compressed down to start.  They can
subsequently be grown out in all different directions, one at a time.
Currently, resizing has no affect on text.

GRAB - The GRAB transitions modify polylines, polygons, and splines by
altering the relative position of a particular vertex on the image (except the
one denoting the image's position) by the relative offsets of the path.  As
opposed to RESIZE, the transitions GRAB1-GRAB7 alter only one particular
vertex in the image's definition. Think of grabbing that vertex and swinging
it around while all the other points stay anchored.  GRABs have no effect on
other images.

SHUFFLE - This transition type only affects BITMAP image type.  For
each offset in the path argument, the next 2-d bitmap in the image's
series will be displayed.  The series is considered circular, i.e.,
the successor of the last bitmap is the first bitmap.  This transition
is useful to get a certain effect such as a door opening.  To do this,
define a BITMAP image with a series of individual bitmaps showing the
various stages of the door opening.  Then run a SHUFFLE transition to
actually show this happen.

RAISE - bring the given {\em image} to the viewing plane closest to the
viewer. The image's position is not changed, only its relative ordering (top
to bottom) with respect to other images.

LOWER - push the given {\em image} to the viewing plane farthest from the
viewer.  The image's position is not changed, only its relative ordering (top
to bottom) with respect to other images.  It will possibly be obscured by
every other image.

ZOOM - have the window zoom in or out on the positional location of
the image argument.  The {\em x} and {\em y} components of the path
determine how the zooming should occur.  They should be between 0.0
and 1.0.  Values close to 1.0 will zoom in very tight.  Values near
0.0 do a slow zoom.  To get a nice slow, smooth zoom in on an object,
use a path with quite a few offsets all of equal ``small'' offset
values, such as $\langle 0.1, 0.1\rangle$.  Positive offset values
zoom the display in, negative offset values pull the display back out.

DELETE - make an image invisible (if not already) and remove it from
any possible later usage.  Any attempts to use this image as a
parameter in subsequent TANGO calls will result in error messages.
Also, pre-existing transitions with this image as a parameter should
not be performed.

REFRESH - redraw the entire animation window, restoring it to a pristine
state.  This is useful when running TANGO in fast mode (non-total
frame by frame refresh) and you would like
to clean up damage to the screen.  A refresh is done for each offset in the
path, i.e., there's no need for more than one offset in the path.  The image
used in this transition does not matter and can be NULL.

DELAY - create a transition which has a null frame for each offset in the
path.  By null frame, we mean that no action or modification is performed.
This transition is helpful when combined with concatenation in order to
produce actions that commence at varying times.  Note that NULL is an
acceptable image for use in the delay transition.


\vspace{1em}
\begin{verbatim}
TANGO_TRANS
TANGOtrans_iterate (trans,num) 
   TANGO_TRANS trans;
   int num;
\end{verbatim}
This routine returns a transition which corresponds to performing the given
{\em trans} {\em num} times, one immediately after the other.  One
example
use of this routine is to make an image repeat a given behavior.

\vspace{1em}
\begin{verbatim}
TANGO_TRANS
TANGOtrans_concatenate (num,t1,t2,t3,t4,t5,t6,t7,t8,t9,t10) 
   int num;
   TANGO_TRANS t1,t2,t3,t4,t5,t6,t7,t8,t9,t10;
\end{verbatim}
This routine returns a transition which corresponds to the concatenation of
the transitions in the {\em t}i parameters. The parameter
{\em num} denotes the number of valid transitions being passed in.  The order
of
the incoming transitions corresponds to the order that they will be
performed
in the concatenation with transition {\em t1} being first.
This one logical transition can then be composed, iterated, etc.,  with other
transitions to achieve various desired animations.

\vspace{1em}
\begin{verbatim}
TANGO_TRANS
TANGOtrans_compose (num,t1,t2,t3,t4,t5,t6,t7,t8,t9,t10) 
   int num;
   TANGO_TRANS t1,t2,t3,t4,t5,t6,t7,t8,t9,t10;
\end{verbatim}
This routine returns a transition which is the composition of the {\em num}
transitions given in the subsequent parameters.  To call the function and
compose 2 transitions, call TANGOtrans\_compose(2,t1,t2).  To compose more
than 10 transitions, you need to call this routine twice with the result
of the first call as a parameter to the second call.
By composition,
we mean the ``concurrent'' execution of all the transitions involved.  When
transitions are composed, individual frames of the transition are combined,
e.g., the first frames, the second frames, etc.  For transitions, each offset
in the path utilized corresponds to one frame.  Transitions of unequal
``length'' can be composed.  Transitions shorter than the longest transition
will simply have null action frames added to their tails.  Consequently, all
the action will start together but will finish according to how many frames
are in the transition.

This routine is very important because it provides the ability to have many
images moving and changing around the screen at the same time.

\vspace{1em}
\begin{verbatim}
void
TANGOtrans_perform (trans) 
   TANGO_TRANS trans;
\end{verbatim}
This routine actually carries out the given transition which has been created
via one of the transition creation routines.  All the involved graphical
images are animated in the ways specified in the transition when it was
created.

\vspace{1em}
\begin{verbatim}
void
TANGOtrans_free (num,t1,t2,t3,t4,t5,t6,t7,t8,t9,t10) 
   int num;
   TANGO_TRANS t1,t2,t3,t4,t5,t6,t7,t8,t9,t10;
\end{verbatim}
This routine frees the space associated with TANGO\_TRANSs.  {\em num}
indicates the number of transitions that are being passed in through the first
{\em t1} to {\em tnum} parameters.  Any number from one to ten is allowable.
The space associated with each of the transitions will be freed. It is a good
idea to call this routine when a transition will no longer be used.

\subsection{Miscellaneous} 
TANGO provides limited input
capabilities.  Currently, it allows the end-user to interactively
select real-value coordinates and image objects in the animation
window.  Data can then be passed back to the driving program through
return values of animation scenes.

TANGO also provides a handle onto all of the valid X colors.  There is
a routine that can be used to load an X color in TANGO's color
database.

TANGO also provides ways for a designer to zoom and/or pan the
display.  There is a routine to set the window boundary coordinates
(also one to inquire the current coordinates).

\vspace{1em}
\begin{verbatim}
int
TANGOinput_coord (x,y) 
   double *x,*y;
\end{verbatim}
This routine prompts the user to select a coordinate from the animation
window.  If a successful choice is made, the routine returns 1 and places the
{\em x} and {\em y} values into the respective parameters.  If for some reason
the selection was not successful (e.g., the mouse selection was made
outside the tango animation window), the routine returns 0 and places 0.0
in both parameters.

\vspace{1em}
\begin{verbatim}
int
TANGOinput_image (image) 
   TANGO_IMAGE *image;
\end{verbatim}
This routine prompts the user to select an image from the animation
window.  The user clicks down the mouse and this routine returns the
uppermost {\bf visible} image object whose bounding box (with a little
padding, hence you can select vertical and horizontal lines) encloses
the point of the mouse selection. If a successful image choice is
made, the routine returns 1 and places the chosen TANGO\_IMAGE into
the {\em image} parameter.  If no image was selected, the routine
returns 0 and places NULL in the parameter.  Be careful with lines;
they often have big bounding boxes.  The raise and lower transitions
are often useful aids for inputting images.

\vspace{1em}
\begin{verbatim}
TANGO_COLOR
TANGOload_color (colorname)
   char *colorname;
\end{verbatim}
This routine loads a color into TANGO's database of colors.  The color
string name should be one from X's color name database (usually in
/usr/lib/X11/rgb.txt on most systems).  The returned TANGO\_COLOR can
then be used in subsequent image creation calls.  Note that the color
cannot be used as a parameter to TANGOpath\_color.  Only one of the
basic 8 colors can be used there.

\vspace{1em}
\begin{verbatim}
int
TANGOset_bgcolor (colorname) 
   char *colorname;
\end{verbatim}
This routine can be used to change the background color of the xtango
display area.  Its effect is immediate.  The default is white.

\vspace{1em}
\begin{verbatim}
int
TANGOset_coord (lx,by,rx,ty) 
   double lx, by, rx, ty;
\end{verbatim}
This routine is used to manually set the window boundary coordinates
which start by default at 0.0 to 1.0 in both directions (origin is at
upper left corner).  This routine then allows you to zoom or pan the
display from with an animation file.  It is advisable to maintain the
same aspect ratio between $x$ and $y$ or weird things may happen.  A
sequence of calls to this routine with gradually tightening window
coordinates will provide a very nice smooth zooming functionality.

\vspace{1em}
\begin{verbatim}
int
TANGOinq_coord (lx,by,rx,ty) 
   double *lx, *by, *rx, *ty;
\end{verbatim}
This routine returns the current value of the window's boundary
coordinates. 

\subsection{TWIST Image and Location Grouping}
These routines provide user-level, high-level, friendly access to the
TANGO calls. 
They primarily consist of routines to create arrays or groups of images and
locations.  (TWIST = Tango's Wonderful Image Synthesis Tool).
It is a form of higher-level macro facility for more convenient animation
design and calls.

\subsubsection{Locations}
The routines provide the capabilities to create one-dimensional and
two-dimensional arrays of locations, as well as a binary tree of locations.

\vspace{1em}
\begin{verbatim}
void
TWISTcreate_loc_array (assoc,id,num,horiz,x,y,spacing) 
   char *assoc;
   int id;
   int num;
   int horiz;
   double x,y;
   double spacing; 
\end{verbatim} 
This routine creates a
one-dimensional line of TANGO\_LOCs.  The locations can be accessed
through associations.  If {\em assoc} is non-null, the given string is
used as the association name used to set up the locations.  If it is
null, the association is ``ID''.  {\em id} is the identifier used as a
parameter to the association.  {\em num} provides the number of
locations to create.  If {\em horiz} is 1, the locations are set up in
a horizontal row; if {\em horiz} is 0, a vertical column is created.
{\em x} and {\em y} denote the starting location for the line, topmost
location if vertical, and leftmost if horizontal.  {\em spacing} gives
the amount of space to place between locations.  

Note: to reference individual locations after this routine has been
called, use the call ASSOCretrieve("ID", id, index\#) assuming that no
association is given.

\vspace{1em} 
\begin{verbatim} 
void 
TWISTcreate_2d_loc_array (assoc,id,rows,cols,x,y,xspacing,yspacing)
   char *assoc;
   int id;
   int rows,cols;
   double x,y;
   double xspacing,yspacing;
\end{verbatim}
This routine creates a two-dimensional grid of TANGO\_LOCs.  If {\em assoc} is
NULL, the association ``ID3'' is utilized.  If {\em assoc} is provided, it is
used.  {\em id} is the identifier used as the first parameter to the
association set up for the locations.  {\em rows} and {\em cols} indicate the
number of rows and columns to be created.  {\em x} and {\em y} denote the
location of the upper left corner of the grid.  {\em xspacing} and
{\em yspacing} provide the amount of space to place between locations.
.pp
Note:  to reference a particular location after the routine has been used,
make the call ASSOCretrieve("ID3", id, row, col).  They start at 0.

\vspace{1em}
\begin{verbatim}
void
TWISTcreate_bintree (id,lx,by,rx,ty,edgelevels) 
   int id;
   double lx,by,rx,ty;
   int edgelevels;
\end{verbatim}
This routine lays out a binary tree within the bounding box designated by
{\em lx, by, rx, ty}.  The tree will contain the given number of edge levels
(e.g., "1" designates a root and two children).  All this routine does is to
create a TANGO\_LOC where each of the nodes should be placed (top-most point).
It does NOT do any image creation.  This routine also sets up associations
that are helpful for accessing the TANGO\_LOCs.  Each location can be accessed
through the ASSOCretrieve("NODE",id,i,j) call, where {\em id} is the parameter
passed into this routine, {\em i} is the depth of the node (root is depth 0),
and {\em j} is the ordering number of the node across its level.  (Leftmost
nodes start at position 0.)  For example, {\em i}=2 and {\em j}=2 retrieves
the location that is the left child of the right child of the root.  Locations
can also also be accessed by the call ASSOCretrieve("NUMBER",id,i) call, where
{\em i} is the numeric placement of a node, in an order from
top-to-bottom in depth, and then left-to-right across the tree.  The
root is node 0; its left child is 1 and its right child is 2, etc. 
This routine
also sets up associations for accessing child and parent nodes.  Calling
ASSOCretrieve("PARENT",id,loc) returns the location that is the parent of
{\em loc} ({\em loc} is a TANGO\_LOC).  The calls
ASSOCretrieve("LCHILD",id,loc) and ASSOCretrieve("RCHILD",id,loc) return the
left and right child locations for {\em loc} respectively.

This routine allows you to draw the style of node and edge that you like.  It
merely lays out the positions (top-center justified) where they should go.
Note that by providing different {\em id}s, you can get multiple trees in the
same animation window.  Also note that the leaf node locations will reside on
the horizontal line with {\em y}={\em by}.  Trying to retrieve the child of a
leaf or the parent of the root returns NULL.

\subsubsection{Images}
These routines allow the creation of arrays or graphs of images.
\vspace{1em}
\begin{verbatim}
void
TWISTcreate_image_array (as,id,n,ty,hrz,jst,x,y,xs,xqu,ys,yqu,sp,vis,col,fill,wid,sty) 
   char *as;
   int id;
   int n;
   TANGO_IMAGE_TYPE ty;
   int hrz;
   int jst;
   double x,y;
   double xs[];
   double xqu;
   double ys[];
   double yqu;
   double sp;
   int vis;
   int col;
   double fill;
   double wid;
   double sty;
\end{verbatim}
This routine creates a one-dimensional array (row or column) of
images.  The images can be accessed through associations.  If {\em as}
is NULL, the default association ``ID'' is utilized.  An image can be
``gotten'' by its id and index number.  If an {\em as} string is
provided, it is used to make this identifying association.  {\em id}
identifies the images to the association package.  {\em n} gives the
number of images in the array.  All the images are of the given
TANGO\_IMAGE\_TYPE {\em ty}.  This type can be
TANGO\_IMAGE\_TYPE\_LINE, TANGO\_IMAGE\_TYPE\_RECTANGLE, or
TANGO\_IMAGE\_TYPE\_CIRCLE.  If the parameter {\em hrz} is 1, a
horizontal row of images is created.  If the parameter is 0, a
vertical column is created. {\em jst} provides the justification of
the images.  If a row is chosen, the images are bottom-justified when
{\em jst} is 0, middle-justified if 1, and top-justified if 2.  If a
vertical column is chosen, the images are left-justified if {\em jst}
is 0, middle-justified if 1, and right-justified if 2.  {\em x} and
{\em y} provide the starting location of the array of images.  For a
row, the {\em x} value denotes the left starting location, and the
{\em y} value is the bottom, center, or top of the images depending on
their justification.  If a column is chosen, the {\em y} value denotes
the top of the first image in the column, and the {\em x} value
denotes the left, center, or right alignment position of the images.
The {\em (x,y)} and {\em qu} parameters for x and y allow the images
to be varying sizes.  If the x's pointer parameter is NULL, all images
are constant in the appropriate dimension with the given {\em qu}
value denoting the size.  To get varying heights of images, the {\em
(x,y)s} parameter should point to an array (of length equal to {\em
n}) of real numbers (ranging between 0.0 and 1.0).  When the images
are drawn, each image's height will be its indexed spot in {\em ys}
times the {\em yqu} argument.  Widths are determined by the {\em x}
arguments.  For circles, only the {\em x} parameters matter.  They
affect the circles' radii.  {\em sp} provides the amount of space
between images along the main axis direction.  If {\em vis} is 0, the
images are initially invisible.  (1 makes them visible.)  {\em col}
indicates the color that the images will be.  {\em fill} should be in
the range 0.0 (outline) to 1.0 (solid color fill).  {\em wid} and {\em
sty} apply to lines.  They should be in the range 0.0 to 1.0.

\vspace{1em}
\begin{verbatim}
void
TWISTcreate_graph (assoc,id,num,loc,adj,im_type,size,fill) 
   char *assoc;
   int id;
   int num;
   TANGO_LOC loc[];
   int adj[][50];
   TANGO_IMAGE_TYPE im_type;
   double size;
   double fill;
\end{verbatim}
This routine draws the image of a graph given the locations of its vertices
and its adjacency matrix.  The graph is drawn in the classical manner with
each vertex represented by a circular or square node, with connected vertices
joined by a line. {\em id} and {\em num} identify the graph and the
number of vertices.  {\em loc} is an array of TANGO\_LOCs specifying the
locations of the vertices.  {\em adj} is the standard adjacency matrix
denoting which vertices are connected by an edge.  1 indicates connected, 0
indicates not connected.  {\em im\_type} gives the TANGO\_IMAGE\_TYPE to make
each vertex's graphical image.  This can be either TANGO\_IMAGE\_TYPE\_RECTANGLE
or TANGO\_IMAGE\_TYPE\_CIRCLE.  {\em size} gives the size of each image (side of
the square for a rectangle, and radius for a circle.)  {\em fill} applies to
the image nodes, and should be a value in the range 0.0 to 1.0. (Graphs are
always drawn in black color.)  The following associations are set up
by this routine:  If {\em assoc} is null, under the association ``ID'' with
parameters {\em id} and index, each TANGO\_LOC can be accessed.  (If an
{\em assoc} is provided, it is used there.)  Also, given the inquiry
ASSOCretrieve("IMAGE\_AT",{\em id},tango\_loc), the particular TANGO\_IMAGES can
be acquired.  Finally, to access the TANGO\_IMAGE which is the line between two
vertices, make the call
ASSOCretrieve("EDGE\_BETWEEN",{\em id},from\_index,to\_index). (The from and to
ordering does not matter.  Either way will retrieve the image.)



\subsection{Associations}
These routines provide the intermediary between the data of a program and the
animation objects associated with this data.  An association is defined by a
name and the number of parameters to the association.  The number of
parameters can be from 0 to 5.  The parameters or data object stored
with an association can be any valid 32 bit quantity such as an integer or a
pointer.  We have indicated them as integers in the following routine
definitions.

\vspace{1em}
\begin{verbatim}
void
ASSOCinit () 
\end{verbatim}
This routine initializes the ASSOC package and is called automatically by all
the other ASSOC routines if the package has not already been initialized.
It makes the reserved associations ``ID'' (2 parameters), ``ID3''
(3 parameters), and ``IMAGE\_AT'' (2 parameters).

\vspace{1em}
\begin{verbatim}
void
ASSOCmake (name,params) 
   char *name;
   int params;
\end{verbatim}
This routine creates an association for later use.  Note that an association
must be created with this routine before subsequent calls to ASSOCstore and
ASSOCretrieve can be issued for the association.  {\em name} provides the name
of the association, and {\em params} provides the number of indexing
parameters needed to reference a record of the association.

\vspace{1em}
\begin{verbatim}
void
ASSOCstore (name,p1,p2,p3,p4,p5,p6) 
   char *name;
   int p1,p2,p3,p4,p5,p6;
\end{verbatim}
This routine allows a record to be stored under a particular association and
set of indices.  {\em name} gives the name of the association.  The indexing
parameters are sent in {\em p1} through {\em pn} assuming the association was
made with {\em n} parameters.  The record to be stored is in the
{\em n+1}
parameter.  For instance, to store the record 29 under the association "ID"
with indices 12334 and 2, use the call ASSOCstore("ID", 12334, 2, 29).
Note that TANGO\_IMAGEs, TANGO\_LOCs, TANGO\_PATHs, and TANGO\_TRANSs
are valid 32 bit quantities and they can be stored in associations.

\vspace{1em}
\begin{verbatim}
int
ASSOCretrieve (name,p1,p2,p3,p4,p5) 
   char *name;
   int p1,p2,p3,p4,p5;
\end{verbatim}
This routine is used to retrieve the data record stored under a given
association and set of indices.  {\em name} provides the name of the
association; the indices are given in the following parameters.  To retrieve
the record stored in the example above in ASSOCstore, make the call
ASSOCretrieve("ID", 12334, 2). Typically, the value returned must be cast into
its appropriate format such as a pointer to a character or an TANGO\_IMAGE for
example.  Note that if no object has been stored under the provided association
parameters, the routine returns NULL and prints out an error message
(usually long and gory).

\vspace{1em}
\begin{verbatim}
int
ASSOCmember (name,p1,p2,p3,p4,p5) 
   char *name;
   int p1,p2,p3,p4,p5;
\end{verbatim}
This routine is used to determine if a particular association and set
of parameters has a data object stored under it.  {\em name} provides the
name of the association; the indices are given in the following parameters.  
The routine returns 1 if there is a data object stored, and 0 if no
association store has been done.  Basically, this routine provides a
safe way to check for data object existence, in the case of uncertainty,
before calling ASSOCretrieve.

\vspace{1em}
\begin{verbatim}
int
ASSOCdelete (name,p1,p2,p3,p4,p5) 
   char *name;
   int p1,p2,p3,p4,p5;
\end{verbatim}
This routine is used to remove a data object from being stored under
an association.  Utilize the exact same parameters as you did to store
the object.  After this routine has been called, subsequent calls to
ASSOCmember will return 0, and ASSOCretrieve returns NULL.

\subsection{TAP Routines}
These routines provide high-level user access to certain TANGO calls.
They consist of routines designed to create transitions for certain
often used actions.  (TAP = TANGO's high-leve Animation Package).
Almost all the routines return a transition, rather than actually performing
it; in this way, the transition can be manipulated outside the corresponding
TAP routine (i.e. the transitions can be concatenated, composed, etc., if
needed).  TAP, like TWIST, is a form of higher-level macro facility for easier
and faster animation implementation; however, it is not absolutely
necessary to use the routines, as any of the transitions could be created
by the user.

\vspace{1em}
\begin{verbatim}
void
TAPshow (image)
   TANGO_IMAGE image;
\end{verbatim}
This routine simply displays {\em image} (i.e. uses a delay transition
to display the object).  Since an image will not appear until a subsequent
animation occurs, this routine is useful for making sure that an image is
properly displayed.

\vspace{1em}
\begin{verbatim}
TANGO_TRANS
TAPfill (image)
   TANGO_IMAGE image;
\end{verbatim}
This routine creates the transition needed to fill the given image to a
solid color.  The function then returns the transition.

\vspace{1em}
\begin{verbatim}
TANGO_TRANS
TAPcolor (image, color)
   TANGO_IMAGE image;
   TANGO_COLOR color;
\end{verbatim}
This routine creates the necessary transition to change {\em image} to the
given color and returns this transition.

\vspace{1em}
\begin{verbatim}
TANGO_TRANS
TAPvis_toggle(image)
   TANGO_IMAGE image;
\end{verbatim}
This routine creates the transition required to toggle the visibility of
{\em image} and returns this transition.

\vspace{1em}
\begin{verbatim}
TANGO_TRANS
TAPjump(image, part, location)
   TANGO_IMAGE image;
   TANGO_PART_TYPE part;
   TANGO_LOC location;
\end{verbatim}
This routine sets up a transition to move {\em image} to the given location
using the specified part as a ``handle.''  The transition uses a path of
length one, so essentially, the image will appear to ``jump'' from its current
location to the new location.  The necessary transition is returned.

\vspace{1em}
\begin{verbatim}
TANGO_TRANS
TAPmove(image, part, location)
   TANGO_IMAGE image;
   TANGO_PART_TYPE part;
   TANGO_LOC location;
\end{verbatim}
This routine sets up a transition to move {\em image} from the location
of its indicated {\em part} to the given location .  This routine is similar
to TAPjump; however, instead of using a path of length one to create the
transition, it uses TANGOpath\_motion to create the transition, which means
that the path length is 20.  The required transition is returned.

\vspace{1em}
\begin{verbatim}
TANGO_TRANS
TAPtraverse(image, part, location, motion)
   TANGO_IMAGE image;
   TANGO_PART_TYPE part;
   TANGO_LOC location;
   TANGO_PATH_TYPE motion;
\end{verbatim}
This routine creates the transition necessary to move an object (using the
specified part of the object) from its current position to {\em location}.
The type of path that the object follows is given by the parameter
{\em motion}; {\em motion} is simply a path type, i.e. it is of the same
form as the path types for the TANGOpath\_motion routine.

\vspace{1em}
\begin{verbatim}
TANGO_TRANS
TAPswitch(image1, image2)
   TANGO_IMAGE image1, image2;
\end{verbatim}
This routine swaps two images; that is, {\em image1} will acquire
{\em image2's} location and vice versa.  The path that each image follows
has a length of one, so the two images will appear to ``jump'' to their
new locations.

\vspace{1em}
\begin{verbatim}
TANGO_TRANS
TAPexchange(image1, image2)
   TANGO_IMAGE image1;
   TANGO_IMAGE image2;
\end{verbatim}
This routine exchanges the position of two images.  It is similar to TAPswitch;
however, it moves each image over a path of length 20 using TANGOpath\_motion.

\vspace{1em}
\begin{verbatim}
TANGO_TRANS
TAPflash(image, num_flash)
   TANGO_IMAGE image;
   int num_flash;
\end{verbatim}
This routine causes an object to toggle between visible and invisible for the
given number of times.  This makes an object appear to flash.




\end{document}
