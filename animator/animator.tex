\documentstyle[11pt]{article}

\textheight 9in
\textwidth 6in
\topmargin -0.5in
\oddsidemargin 0.25in
\evensidemargin 0.25in

\newcommand{\quest}{\vspace{3ex}\noindent}

\begin{document}

\vspace*{.3in}

\begin{center}
{\Large\bf Animator Program}
\end{center}

This document describes the Animator program which provides an
interpreted, interactive animation front-end to XTango.  The Animator
program simply reads an ascii file, one command per line.  This is
beneficial because you can have the output of any program, be it
Pascal, C, Modula-2, etc., drive an animation.  I have used this tool
in an undergraduate algorithms class.  In addition to implementing an
algorithm, students can develop an animation of it just by judiciously
placing print statements into their program.

Animations developed with this system will be carried out in a
real-valued coordinate system that originally runs from 0.0 to 1.0
from left-to-right and from 0.0 to 1.0 from bottom-to-top.  Note,
however, that the coordinate system is infinite in all directions.
You will create and place graphical objects within the coordinate
system, and then move them, change their color, visibility, fill,
etc., in order to depict the operations and actions of a computer
algorithm.

The format for the individual commands is
described below.  You can run this program interactively by piping the
output of your program to it, e.g., 
\begin{quote}
{\tt \% yourprog | animator}
\end{quote}
If your program was developed on another system, you can simply import
its textual output as a file and read that file as input, e.g.,
\begin{quote}
{\tt \% animator < outfile}
\end{quote}
What you need to do is augment your implementation of the algorithm
under study by a set of output (e.g., printf, writeln) statements.
The statements should be placed in the program at the appropriate
positions to provide a trace or depiction of what your program is
doing. 

Below we summarize the different commands that exist within the
system.  If there's a command you would like added, just ask.  To
begin this section, we describe the commands in general.  

Each command begins with a unique one word string.  Make sure that you
spell the strings correctly.  Each graphical object that you create
should be designated by a unique integer id.  You will need to use
that id in subsequent commands that move, color, alter, etc., the
object.  In essence, the id is a handle onto the object.  Most of the
commands and their parameters should be self-explanatory.  Arguments
named {\tt steps} and {\tt centered} are of integer type.  Arguments
named {\tt xpos, ypos, xsize, ysize, radius, lx, by, rx, ty} are real
or floating point numbers.  Make sure that they include a decimal
point (typing 0 instead of 0.0, for example, will cause an error).
The argument {\tt fillval} should be one of the following strings:
{\tt outline}, {\tt half} or {\tt solid}.  The argument {\tt widthval}
should be one of the following strings: {\tt thin}, {\tt medthick}, or
{\tt thick}.  The parameter {\tt
colorval} can be any color string name from the file {\tt
/usr/lib/X11/rgb.txt} (one word, no blanks allowed).  Note that the
{\tt color} command only supports a subset of all these possible
colors.

Note that any line with a per cent character (\%) in column 1 is
interpreted as an instructional comment---no command is carried out.

The animator will print warning messages if you try to do erroneous
actions like reuse an integer ID, try to delete or move an invalid ID,
and so on.  You can turn these warnings off by using the -O command
line flag.

\quest
\tt comment A trailing string
\hspace*{3ex} \rm This command simply prints out any text following the
``comment'' identifier to the shell in which the animator was invoked.

\quest
\tt bg colorval\\
\hspace*{3ex} \rm Change the background to the given color.  The
default starter is white. 

\quest
\tt coords lx by rx ty\\
\hspace*{3ex} \rm Change the displayed coordinates to the given
values.  You can use repeated applications of this command to pan or
zoom the animation view.

\quest
\tt delay steps\\
\hspace*{3ex} \rm Generate the given number of animation frames with
no changes in them.

\quest
\tt line id xpos ypos xsize ysize colorval widthval\\
\hspace*{3ex} \rm Create a line with one endpoint at the given
position and of the given size.

\quest
\tt rectangle id xpos ypos xsize ysize colorval fillval\\
\hspace*{3ex} \rm Create a rectangle with lower left corner at the
given position and of the given size (size must be positive).

\quest
\tt circle id xpos ypos radius colorval fillval\\
\hspace*{3ex} \rm Create a circle centered at the given position.

\quest
\tt triangle id v1x v1y v2x v2y v3x v3y colorval fillval\\
\hspace*{3ex} \rm Create a triangle whose three vertices are located
at the given three coordinates.  Note that triangles are moved (for
move, jump, and exchange commands) relative to the center of their
bounding box.

\quest
\tt text id xpos ypos centered colorval string\\
\hspace*{3ex} \rm Create text with lower left corner at the given
position if {\tt centered} is 0.  If {\tt centered} is 1, the position
arguments denote the place where the center of the text is put.  
The text string is allowed to have blank spaces included in it but you
should make sure it includes at least one non-blank character.

\quest
\tt bigtext id xpos ypos centered colorval string\\
\hspace*{3ex} \rm This works just like the text command except that
this text is in a much larger font.  

\quest
\tt flextext id xpos ypos centered colorval fontname string\\
\hspace*{3ex} \rm This is the flexible text command, and it works just
like the text command except that you can explicitly specify the font
(string name) that you want to use.  Any valid X11 font is allowable.

\quest
\tt move id xpos ypos\\
\hspace*{3ex} \rm Smoothly move, via a sequence of intermediate steps, the object with the given id to the specified
position.

\quest
\tt moverelative id xdelta ydelta\\
\hspace*{3ex} \rm Smoothly move, via a sequence of intermediate steps,
the object with the given id by the given relative distance.

\quest
\tt moveto id id\\
\hspace*{3ex} \rm Smoothly move, via a sequence of intermediate steps,
the object with the first id to the current position of the object
with the second id.

\quest
\tt jump id xpos ypos\\
\hspace*{3ex} \rm Move the object with the given id to the designated
position in a one frame jump.

\quest
\tt jumprelative id xdelta ydelta\\
\hspace*{3ex} \rm Move the object with the given id by the provided
relative distance in one jump.

\quest
\tt jumpto id id\\
\hspace*{3ex} \rm Move the object with the given id to the current
position of the object with the second id in a one frame jump.

\quest
\tt color id colorval\\
\hspace*{3ex} \rm Change the color of the object with the given id to
the specified color value.  Only the colors white, black, red, green,
blue, orange, maroon, and yellow are valid for this command.

\quest
\tt delete id\\
\hspace*{3ex} \rm Permanently remove the object with the given id from
the display, and remove any association of this id number with the object.

\quest
\tt fill id fillval\\
\hspace*{3ex} \rm Change the object with the given id to the
designated fill value.  This has no effect on lines and text.

\quest
\tt vis id\\
\hspace*{3ex} \rm  Toggle the visibility of the object with the given
id.

\quest
\tt lower id\\
\hspace*{3ex} \rm  Push the object with the given id backward to the viewing
plane farthest from the viewer.

\quest
\tt raise id\\
\hspace*{3ex} \rm  Pop the object with the given id forward to the
viewing plane closest to the viewer.

\quest
\tt exchangepos id id\\
\hspace*{3ex} \rm Make the two objects specified by the given ids
smoothly exchange positions.

\quest
\tt switchpos id id\\
\hspace*{3ex} \rm Make the two objects specified by the given ids
exchange positions in one instantaneous jump.

\quest
\tt swapid id id\\
\hspace*{3ex} \rm Exchange the ids used to designate the two given
objects. 

\end{document}
